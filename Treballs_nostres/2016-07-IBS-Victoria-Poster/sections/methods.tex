\section{Methods for pattern selection}

\subsection{Based on Conditional Mutual Information}

When studying methylation we are faced with two main questions:
  \begin{enumerate}
  \item Which genes exhibit an L-shape, and 
  \item what is the optimal threshold for binarizing
methylation data for each L-shape gene.
  \end{enumerate}
To determine whether methylation and expression of a gene exhibit an L-shape,
compute the conditional Mutual Information (MI) for different choices of threshold
to binarize the methylation data.


If we consider the continuous valued methylation and expression data as two random variables
$X$ and $Y$, and denote a nominal threshold as $t$, the conditional MI can be written as a
weighted sum of MIs on the two sides of the threshold.
\[
\mathit{cMI}(t)=I(X,Y|X>t)P(X>t) + I(X,Y|X\le t)P(X\le t)
\]

%When $t$ is $0$ or $1$, $\mathit{cMI}$ equals to the mutual information derived 
%from all data points.

For an L-shape gene, as $t$ moves from 0 to 1, $\mathit{cMI}(t)$ first decreases and then
increases, and its value approaches zero when $t$ coincides with the reflection point. 
  
The ratio $r=\frac{\min\{\mathit{cMI}(t)\}}{\mathit{cMI}(0)}$ for an L-shape gene is small, 
and $t^{\ast} = \mathrm{argmin}\{ \mathit{cMI}(t) \}$ is the \textbf{optimal threshold} for 
dichotomizing the methylation data of this gene.


To estimate the MI terms we use a kernel-based estimator, which 
constructs a joint probability distribution by applying a Gaussian kernel to each data point:
\[
I(X,Y) = \frac 1M \sum_{i=1}^M \log\frac{M\sum_{j=1}^M e^{-\frac{1}{2h^2}((x_i-x_j)^2+(y_i-y_j)^2)}}{%
                                      \sum_{j=1}^M e^{-\frac{1}{2h^2}(x_i-x_j)^2} \sum_{j=1}^M e^{-\frac{1}{2h^2}(y_i-y_j)^2}}
\]
where $h$ is a tuning parameter for the kernel width and empirically set $h=0.3$.
  

% i and j are indices for samples.
% In our analysis, we normalize the expression data to zero mean.

\subsection{Based on Spline regression}


%We implemented regression based on  $B$-splines because they are particularly efficient due to the block-diagonal basis matrices that result.

Let 

 $\varsigma=\lbrace t_1 < \cdots < t_N \rbrace$ non decreasing  knot sequence 
 $\left[ t_m,t_{m+1} \right)$ half open interval
 $B_{mp}$ $p$-th order polynomial (degree $p-1$) with finite support over the interval and 0 everywhere else so that  $\sum_{m=1}^{N-p}B_{mp}(x)=1$
 then  $s(x)=\sum_{m=1}^{N-p}B_{mp}(x)c_m$ 


To represent the curve we set $y_{ij}=s(x_{ij})$, so $\mathbf{y}_i=\mathbf{B}_i\mathbf{c}$
with
\begin{description}
\item $\mathbf{B}_i =\left[ B_{1p}\mathbf{x}_i,B_{2p}\mathbf{x}_i,\dots,B_{Lp}\mathbf{x}_i \right]$ the spline basis matrix 
\item $\mathbf{c}$ the vector of spline coefficients
\end{description}

\textbf{Algorithm}
\begin {enumerate}
\item Selection of the genes with a negative significant correlation
\item Fit cubic regression splines
\item Data to cluster: splines coefficients
\item Calculation of a distance matrix between genes as $1-\rho$
\item Hierarchical clustering 
\end{enumerate}
