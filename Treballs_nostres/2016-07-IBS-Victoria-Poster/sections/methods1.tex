\section{Methods for pattern selection}

\subsection{Based on Conditional Mutual Information}
\begin{itemize}
\item When studying methylation we are faced with two main questions:
  \begin{enumerate}
  \item Which genes exhibit an L-shape, and 
  \item What is the optimal threshold for binarizing
methylation data for each L-shape gene.
  \end{enumerate}
\item Following \cite{Liu} in order to determine whether methylation $X$ and expression $Y$ of a gene exhibit an L--shape, the conditional Mutual Information $cMI(t)$ for different choices of threshold $t$ is computed.
\[
\mathit{cMI}(t)=I(X,Y|X>t)P(X>t) + I(X,Y|X\le t)P(X\le t)
\]
\item If the relation between methylation and expression shows an L-shape  as $t$ moves from 0 to 1, $\mathit{cMI}(t)$ first decreases and then increases, its value approaching zero when $t$ coincides with the reflection point. 
\end{itemize}
 