\subsection{Selection Based on Spline regression}
\begin{itemize}
\item Spline regression \cite{racine} was also considered as a basis for scatterplot clustering applying the following algorithm:
%\item In spline regression a curve $y=s(x)$ is represented as $\mathbf{y}_i=\mathbf{B}_i\mathbf{c}$ where 
%\begin{itemize}
%\item $\mathbf{B}_i =\left[ B_{1p}\mathbf{x}_i,B_{2p}\mathbf{x}_i,\dots,B_{Lp}\mathbf{x}_i \right]$ the spline basis matrix and 
%\item $\mathbf{c}$ is the vector of spline coefficients.
%\end{itemize}

%\item This suggests the following method (and algorithm) for detecting L--shaped genes based on \textbf{Clustering Spline Coefficients}:
\begin {enumerate}
\item Select genes with significant negative correlation.
\item For each selected gene fit a cubic splines regression model.
\item Obtain a distance matrix between all genes using the $1-\rho$ distance computed on spline coefficients.
\item Perform a hierarchical clustering and 
\item Select genes in the \textit{L-shaped cluster(s)}.
\end{enumerate}
\end{itemize}
